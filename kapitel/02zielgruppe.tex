\chapter{Zielgruppe}\label{ch:zielgruppe}
Die Zielgruppe des Video-Reviews wird primär aus aktiven Gamern und Streamern gebildet.
Diese Personengruppe zeichnet sich durch ein ausgeprägtes Interesse an Videospielen, insbesondere im Bereich der Action-\glspl{rpg} und Soulslike-Titel, aus.

\section{Primäre Zielgruppe}
Als Kernzielgruppe werden Personen definiert, die regelmäßig Videospiele konsumieren und aktiv nach Informationen über neue oder weniger bekannte Indie-Titel suchen.
Diese Gruppe wird in folgende Segmente unterteilt: Soulslike-Enthusiasten sind Spieler, die bereits Erfahrungen mit bekannten Genre-Vertretern wie \textit{Dark Souls} oder \textit{Elden Ring} gesammelt haben und an alternativen Interpretationen des Genres interessiert sind.
Indie-Game-Interessierte sind Personen, die gezielt nach ungewöhnlichen und experimentellen Spielen außerhalb des \gls{aaa}-Mainstreams suchen und Wert auf einzigartige Artstyles und innovative Konzepte legen.
Content Creator und Streamer sind Gaming-Streamer und YouTuber, die nach interessantem Content für ihre Kanäle suchen und besonders an Spielen mit hohem Unterhaltungswert und Diskussionspotenzial interessiert sind.

\section{Sekundäre Zielgruppe}
Als sekundäre Zielgruppe werden Personen identifiziert, die zwar nicht zur Kernzielgruppe gehören, aber dennoch durch das Format angesprochen werden können.
Casual Gamer sind Spieler, die nur gelegentlich spielen, aber durch den humorvollen Ansatz des Reviews und die unterhaltsame Präsentation angezogen werden.
Retro-Gaming-Fans sind Personen, die sich für den retro-artigen Grafikstil und den nostalgischen \gls{crt}-Filter-Look interessieren.

\section{Zielgruppengerechte Ansprache}
Die Ansprache der Zielgruppe erfolgt durch eine Kombination aus fachlich fundierter Analyse und unterhaltsamer Comedy-Präsentation.
Es wird davon ausgegangen, dass die Zuschauer bereits über Grundkenntnisse in Gaming-Begrifflichkeiten verfügen, jedoch werden komplexere Konzepte wie das Soulslike-Genre explizit erklärt.
Der Tonfall wird bewusst locker und authentisch gehalten, um die Community-Nähe zu wahren, die für Gaming-Content charakteristisch ist.
