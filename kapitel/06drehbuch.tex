\chapter{Drehbuch}\label{ch:drehbuch}

Das Drehbuch wird in elf klar definierte Szenen unterteilt, die zusammen eine Gesamtlaufzeit von circa 7:00 bis 8:00 Minuten ergeben. Jede Szene wird hinsichtlich Inhalt, visueller Umsetzung und Audio-Kommentar detailliert spezifiziert.

\section{Szenenübersicht}

\subsection{Szene 1: Intro (0:00-0:20)}

\textbf{Titel:} INTRO

\textbf{Inhalt:} Ein energetischer Einstieg ins Review wird geschaffen, bei dem das Spiel und seine Besonderheiten vorgestellt werden.

\textbf{Visuelle Umsetzung:} Als energetischer Einstieg wird eine dynamische Intro-Sequenz gezeigt.

\textbf{Audio-Kommentar:} „Was passiert, wenn ihr Soulslikes nehmt, einen CRT-Filter drüberkippt und das Ganze aussehen lasst wie ein Fiebertraum aus den 90ern? Richtig – White Lavender! Heute schauen wir uns dieses bunte Indie-Soulslike an, das beweist, dass man nicht immer düster und ernst sein muss, um schwer zu sein. Spoiler: Es ist chaotisch, es ist bunt, und ja – es hat Bugs. Aber lohnt es sich trotzdem? Let's go!"

\subsection{Szene 2: Soulslikes (0:20-0:45)}

\textbf{Titel:} SOULSLIKES

\textbf{Inhalt:} Eine theoretische Erklärung mit Inhalten, was Soulslikes ausmacht, wird präsentiert.

\textbf{Visuelle Umsetzung:} Als \gls{glos:broll} werden Ausschnitte aus \textit{Dark Souls}, \textit{Elden Ring} und anderen Genre-Vertretern gezeigt.

\textbf{Audio-Kommentar:} „Für die, die noch nie ein Soulslike gespielt haben – hier ein Crash-Course: Soulslikes sind Action-\glspl{rpg}, die für ihre knallharte Schwierigkeit berüchtigt sind. Ihr seid ein Kämpfer in einer Welt voller Monster und Gefahren – das Ziel? Überleben und stärker werden. Präzises Timing, Ausdauer-Management, Bosskämpfe die euch an den Rand des Wahnsinns treiben, und wenn ihr sterbt? Dann dürft ihr schön zurücklaufen und eure Seelen einsammeln... falls ihr es bis dahin schafft. Die bekanntesten Vertreter? Dark Souls, Elden Ring, Bloodborne – Spiele die legendär dafür sind, dass Gamer ihre Controller an die Wand werfen. White Lavender nimmt diese Formel und dreht den Weird-Regler auf Maximum."

\subsection{Szene 3: Grafik (0:45-1:15)}

\textbf{Titel:} GRAFIK

\textbf{Inhalt:} Der super einzigartige retro-artige Grafikstil wird präsentiert: sehr bunt, mit \gls{crt}-Filter, lustigem Charakterdesign und sichtbarer Rüstung am Charakter.

\textbf{Visuelle Umsetzung:} Gameplay von \textit{White Lavender} wird mit Fokus auf die Visuals gezeigt. An die Charakter-Füße wird herangezoomt.

\textbf{Audio-Kommentar:} „Optisch ist das Spiel einfach nicht so das, was man vom Genre kennt. Aber im guten Sinne! Der retro-artige Grafikstil mit dem \gls{crt}-Filter sieht aus, als würde man auf einem alten Röhrenfernseher zocken. Alles ist super bunt und poppig – das komplette Gegenteil von den düsteren Souls-Spielen, die am bekanntesten in dem Genre sind. Das Charakterdesign ist absolut goofy und liebenswert. Und das Beste: Eure Rüstung ist tatsächlich am Charakter sichtbar! Fashion Souls könnte man schon fast sagen. Aber schaut euch mal diese Füße an... die kleben förmlich am Boden. Das Movement ist ebenfalls... nun ja, auch echt speziell. Aber dazu kommen wir jetzt."

\subsection{Szene 4: Movement (1:15-1:35)}

\textbf{Titel:} MOVEMENT

\textbf{Inhalt:} Das Movement-System des Spiels wird vorgestellt.

\textbf{Visuelle Umsetzung:} Ein Movement-Showcase wird präsentiert, bei dem Movement und Fähigkeiten am Anfang des Spiels gezeigt werden.

\textbf{Audio-Kommentar:} „Also, sprechen wir über Movement. Am Anfang fühlt sich euer Charakter an wie... naja, wie jemand der in Honig watet. Ihr habt die Basic-Bewegungen: Laufen, Rollen, Springen – aber es fühlt sich etwas schwerfällig an. Die Füße haben wirklich diesen 'klebe-am-Boden-Effekt', was dem Ganzen einen ungewollt komischen Touch gibt. Aber hey, sobald ihr ein paar Fähigkeiten freischaltet und euch an die Physik gewöhnt habt, wird's besser. Es ist anders, aber man gewöhnt sich dran."

\subsection{Szene 5: Enemies (1:35-2:00)}

\textbf{Titel:} ENEMIES

\textbf{Inhalt:} Der Combat gegen normale Gegner wird im Detail bewertet.

\textbf{Visuelle Umsetzung:} Kampfszenen gegen normale Gegner werden gezeigt, Hitbox-Probleme werden demonstriert und verschiedene Waffen werden präsentiert.

\textbf{Audio-Kommentar:} „Der Combat gegen normale Gegner macht grundsätzlich Spaß – aber er hat leider seine Macken. Timing-basiertes Angreifen, Ausweichen, Ausdauer-Management – das Soulslike-Package ist vollstens vorhanden. ABER: Gerade bei größeren Waffen merkt man den Animation Lock heftig. Einmal geschwungen, seid ihr committed – und dann kanns tödlich sein. Und dann die Hitboxen... besonders bei kleineren Gegnern gehen eure Angriffe manchmal einfach daneben. Das ist frustrierend, wenn ihr wisst, dass der Hit eigentlich sitzen sollte. Aber positiv ist: Das Spiel fördert echt das Waffenexperimentieren! Verschiedene Gebiete, verschiedene Gegnertypen – da lohnt es sich, mehrere Waffen auszuprobieren. Die Waffenvielfalt ist cool, von Standard bis komplett abgedreht."

\subsection{Szene 6: Bosse (2:00-2:30)}

\textbf{Titel:} BOSSE

\textbf{Inhalt:} Das Bossdesign und die Schwierigkeit des Spiels werden bewertet.

\textbf{Visuelle Umsetzung:} Boss-Kampf-Footage wird präsentiert, einschließlich des Endboss-Kampfes und Kampfszenen gegen verschiedene Bosse.

\textbf{Audio-Kommentar:} „Und dann kommen wir zum Herzstück eines jeden Soulslikes, den Bossen... ein zweischneidiges Schwert in diesem Spiel. Das Bossdesign ist kreativ und jeder Boss sieht unique aus. ABER: Die Patterns sind manchmal echt simpel und langweilig. Nach ein, zwei Versuchen habt ihr die meisten Bosse durchschaut. Der Endboss ist im Vergleich zu den anderen Bossen extrem schwach. Wir haben sogar diesen Bug gefunden: [PAUSE] Wenn man direkt unter ihm steht, kann er euch nicht treffen. Da steht man dann und wundert sich... [PAUSE] Die Schwierigkeit ist also sehr inkonsistent – manche Bosse machen euch das Leben zur Hölle und andere macht ihr im Schlaf."

\subsection{Szene 7: Erkunden (2:25-2:45)}

\textbf{Titel:} ERKUNDEN

\textbf{Inhalt:} Das Erlebnis beim Erkunden der Open World wird bewertet.

\textbf{Visuelle Umsetzung:} Open World Exploration wird gezeigt, Portal-Reisen werden demonstriert, das Aufheben von Items wird präsentiert, und verschiedene Gebiete werden durch Portale bereist. Verschiedene Items, die über die Map auffindbar sind, werden gezeigt.

\textbf{Audio-Kommentar:} „Die Open World zu erkunden macht echt Laune! Ihr reist durch verschiedene Gebiete via Portale – und jedes Gebiet hat seinen eigenen Vibe. Von bunten Wäldern, über dunkle Höhlen bis zu trippy Landschaften ist alles dabei. Überall findet ihr Items, Geheimnisse und versteckte Wege. Es lohnt sich also definitiv, jeden Winkel zu erkunden. [PAUSE] Aber das Gefühl, wenn man einen versteckten Schatz findet? \textit{Chef's kiss} Unbezahlbar"

\subsection{Szene 8: \glspl{npc} (2:45-3:05)}

\textbf{Titel:} \glspl{npc}

\textbf{Inhalt:} Die \gls{npc}-Interaktionen werden bewertet hinsichtlich deren Qualität, Tiefe, Questlines und humorvollem Touch.

\textbf{Visuelle Umsetzung:} \gls{npc}-Gespräche und Szenen beim Reden mit \glspl{npc} werden gezeigt.

\textbf{Audio-Kommentar:} „Die \glspl{npc} in White Lavender sind sogar überraschend charmant! Die Dialoge haben einen humorvollen Touch und die Questlines sind echt interessant. Klar, es ist jetzt keine narrative Meisterleistung wie man es von \gls{aaa}-Titeln kennt, aber die \glspl{npc} haben Persönlichkeit und sorgen für ein paar Lacher während des Gameplays. Insgesamt fühlen sich die Interaktionen lebendiger an als man von einem Indie-Titel vielleicht erwarten würde."

\subsection{Szene 9: Musik (3:05-3:20)}

\textbf{Titel:} MUSIK

\textbf{Inhalt:} Es wird kurz erklärt, was für Musik genutzt wird (Genre) und ob es zum Charakter des Spiels passt.

\textbf{Visuelle Umsetzung:} Hintergrund-Gameplay mit Musik wird präsentiert, Hintergrundgameplay von \textit{White Lavender} läuft.

\textbf{Audio-Kommentar:} „Der Soundtrack passt perfekt zum psychedelischen Vibe des Spiels. Es ist oft diese Mischung aus ambient und retro-synth Sounds, die das Ganze zusammenhält. Zusätzlich sind auch wirklich häufig Banger dabei, die man jetzt nicht so erwartet hat. Die Musik untermalt so die sowohl entspannenden Momente wie das Erkunden in der Welt als auch die sehr aufregenden Momente des Spiel wie die Bosskämpfe."

\subsection{Szene 10: Bugs (3:20-3:45)}

\textbf{Titel:} BUGS

\textbf{Inhalt:} Aktuelle Probleme des Spiels und Fehler beim eigenen Durchlauf werden präsentiert.

\textbf{Visuelle Umsetzung:} Bug-Footage wird gezeigt, spezifische Bug-Beispiele werden demonstriert, der Endboss-Bug wird nochmal gezeigt, spezifische Ausschnitte der Bugs werden präsentiert.

\textbf{Audio-Kommentar:} „Okay, jetzt müssen wir noch über den Elefanten im Raum sprechen: nämlich die Bugs im Spiel. Und oh junge, die Liste ist lang. Wir hatten: Enemy Snapping-Probleme – Gegner teleportieren sich plötzlich, Clipping-Issues und Gegner die in Wänden stecken, Framerate-Drops, und der absolute Kracher: Am Ende unseres Runs ließen sich Gegner GAR NICHT MEHR anvisieren. Also so komplett. Das Target-Lock-System hat einfach aufgegeben. Und wie gesagt, der Endboss kann euch nicht treffen wenn ihr unter ihm steht. Das ist jetzt weniger ein Bug und mehr ein 'haben wir nicht gebalanced'-Problem. Für ein Indie-Spiel ist das nicht ungewöhnlich, aber aktuell müsst ihr mit diesen technischen Problemen leider leben. Die Devs patchen hoffentlich fleißig, [PAUSE, dann leise] Gott bewahre..."

\subsection{Szene 11: Fazit (3:45-4:10)}

\textbf{Titel:} FAZIT

\textbf{Inhalt:} Ein Scoring System und eine Bewertung werden über das Video zusammengefasst.

\textbf{Visuelle Umsetzung:} Eine Zusammenfassung mit Best-of Footage wird präsentiert. Im Hintergrund läuft unscharfes Gameplay. Im Vordergrund werden Grafiken der vergebenen Scorings pro Videosektion gezeigt. Ein Outro wird eingeblendet.

\textbf{Audio-Kommentar:} „White Lavender ist ein ambitioniertes Indie-Soulslike mit einer richtig coolen Idee. Der einzigartige Artstyle und die bunte Welt sind erfrischend anders. Das Waffenexperimentieren macht Spaß und die notwendige Varietät ist da. ABER: Die technischen Probleme und Design-Schwächen ziehen das Erlebnis deutlich runter. Animation Lock, miese Hitboxen, simple Bosspatterns, ein viel zu schwacher Endboss, und Bugs die das Spiel teilweise unspielbar machen – das muss man als Spieler auch erstmal schlucken, wenn man das zu spät nach einem Kauf merkt. Es fühlt sich an wie ein Early Access-Titel der noch ein paar Monate Entwicklung braucht. Das Potential ist da, aber aktuell ist es eher was für die hardcore Indie-Fans, die über technische Probleme hinwegsehen können. Wenn die Devs weiter dran arbeiten sollten, könnte das wirklich eine gute Abwechslung zu den eher bekannten Soulslike-Titeln wie eben Dark Souls oder Elden Ring werden. Aber im aktuellen Zustand? Eher schwierig. Habt ihr White Lavender schon gespielt und was sind eure Erfahrungen mit Soulslike Spielen? Lasst es und gerne wissen und bis zum nächsten Mal!"
