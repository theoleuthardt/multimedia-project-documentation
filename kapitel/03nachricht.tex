\chapter{Zu kommunizierende Nachricht}\label{ch:nachricht}
Die zentrale Botschaft des Video-Reviews wird auf mehreren Ebenen vermittelt.
Es wird sowohl eine Kritikäußerung am Spiel, als auch eine grundlegende Wissensvermittlung über das Soulslike-Genre angestrebt.

\section{Primäre Botschaft}
Als Hauptbotschaft wird vermittelt, dass \textit{White Lavender} ein ambitioniertes Indie-Soulslike mit großem Potential darstellt, das jedoch zum aktuellen Zeitpunkt noch unter erheblichen technischen Problemen und Design-Schwächen leidet.
Es wird eine differenzierte Bewertung kommuniziert, die sowohl die positiven als auch die negativen Aspekte des Spiels transparent darstellt.
Die Kernaussage lautet: Das Spiel bietet einen erfrischend anderen Ansatz im Soulslike-Genre durch seinen einzigartigen visuellen Stil und die bunte Welt, kann aber aufgrund von Bugs, unausgewogener Schwierigkeit und technischen Mängeln aktuell nur bedingt empfohlen werden.

\section{Sekundäre Botschaften}
Neben der Hauptbotschaft werden folgende Teilaspekte kommuniziert.
Es wird grundlegendes Wissen über Soulslike-Spiele vermittelt, indem die charakteristischen Merkmale des Genres erklärt werden.
Dadurch werden auch Zuschauer ohne Soulslike-Erfahrung in die Lage versetzt, die Bewertung nachzuvollziehen.
Es wird kommuniziert, dass technische Probleme bei Indie-Titeln nicht ungewöhnlich sind und dass das kreative Potential des Spiels anerkannt wird.
Die Kritik wird konstruktiv formuliert, ohne die Leistung der Entwickler zu diskreditieren.
Es wird vermittelt, dass eine ehrliche und transparente Bewertung wichtiger ist als blindes Lob.
Die identifizierten Probleme werden klar benannt.
Es wird eine klare Empfehlung ausgesprochen: Das Spiel richtet sich primär an hardcore Indie-Fans, die über technische Probleme hinwegsehen können.
Für die breite Masse wird empfohlen, weitere Entwicklungszyklen abzuwarten.

\section{Tonalität der Nachricht}
Die Nachricht wird in einem humorvollen, aber respektvollen Ton vermittelt.
Es wird darauf geachtet, dass die Comedy-Elemente die sachliche Kritik nicht untergraben, sondern die Botschaft zugänglicher und unterhaltsamer machen.
Die Balance zwischen Unterhaltung und Information wird als wesentliches Element der Kommunikationsstrategie verstanden.
