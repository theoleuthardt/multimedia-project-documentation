\chapter{Technische Entscheidungen}\label{ch:technik}
Die technische Realisierung des Video-Reviews basiert auf professionellen Werkzeugen und Formaten, die eine hohe Qualität und effiziente Zusammenarbeit gewährleisten.

\section{Videobearbeitung}

\subsection{Editor-Wahl}
Als primäres Bearbeitungswerkzeug wird DaVinci Resolve eingesetzt.
Diese Wahl wird durch folgende Faktoren begründet: DaVinci Resolve bietet umfassende Bearbeitungsfunktionen für Schnitt, Color Grading und Effekte, die für ein professionelles Review erforderlich sind.
Die kostenlose Version von DaVinci Resolve bietet bereits einen umfangreichen Funktionsumfang, der für dieses Projekt ausreichend ist.
Das Programm ermöglicht einen effizienten Workflow von der Rohschnitt-Phase über das Color Grading bis zur finalen Ausgabe.

\subsection{Videoformat}
Für die Verarbeitung werden die Containerformate \gls{mkv} und \gls{mov} verwendet.
\gls{mkv} wird bevorzugt für Gameplay-Aufnahmen, Zwischenspeicherungen während der Bearbeitung, hohe Flexibilität bei Codec-Wahl und verlustfreie Qualität.
\gls{mov} wird verwendet für Kompatibilität mit DaVinci Resolve, professionelle Farbbearbeitung und standardisierte Ausgabeformate.

\section{Audiobearbeitung}

\subsection{Aufnahme-Software}
Für die Audioaufnahme wird Audacity eingesetzt.
Diese Open-Source-Software wird aufgrund folgender Eigenschaften gewählt: Audacity ist kostenlos verfügbar und bietet dennoch professionelle Aufnahmemöglichkeiten.
Die Benutzeroberfläche ist intuitiv und ermöglicht schnelle Aufnahme-Sessions ohne komplexe Einrichtung.
Grundlegende Audio-Editing-Funktionen wie Normalisierung, Rauschunterdrückung und \gls{eq} sind integriert.

\subsection{Audioformat-Spezifikationen}
Für die Audioaufnahme werden folgende technische Parameter festgelegt: Die Sample Rate beträgt 48 kHz, das Format ist \gls{flac}, die Kanäle sind Stereo (2.0) und die Bit Depth beträgt 24 Bit.
Die Begründung der Parameter lautet wie folgt: Diese Sample Rate wird als Standard für Videoproduktionen verwendet und gewährleistet optimale Kompatibilität mit dem Videomaterial \cite{ITU03}.
Die verlustfreie Kompression erhält die vollständige Audioqualität, während gleichzeitig Speicherplatz eingespart wird.
Dies ist besonders wichtig für die spätere Bearbeitung und Mixing-Prozesse.
Eine Stereo-Konfiguration wird für Voice-Over-Aufnahmen als ausreichend erachtet und bietet gute räumliche Präsenz.
Die höhere Bit-Tiefe gegenüber Standard-16-Bit bietet einen größeren Dynamikumfang und mehr Headroom für die Nachbearbeitung.

\section{Kollaborations-Infrastruktur}

\subsection{Datenspeicherung}
Für den gemeinsamen Zugriff auf Projekt-Assets wird ein \gls{nas}-Speicher eingesetzt.
Die Vorteile umfassen zentralisierte Speicherung aller Audio- und Videodateien, gleichzeitigen Zugriff für alle Teammitglieder, automatische Backup-Möglichkeiten, Versionskontrolle bei größeren Dateien und die Vermeidung von Duplizierung großer Mediendateien.

\subsection{Skript-Entwicklung}
Für die kollaborative Erstellung des Drehbuchs wird Google Docs verwendet.
Die Begründung umfasst folgende Aspekte: Mehrere Personen können gleichzeitig am Skript arbeiten (Echtzeit-Kollaboration).
Feedback kann direkt im Dokument gegeben werden (Kommentarfunktion).
Änderungen sind nachvollziehbar und können rückgängig gemacht werden (Versionsverlauf).
Der Zugriff von verschiedenen Geräten ist möglich (Plattformunabhängigkeit).
Keine lokale Softwareinstallation ist erforderlich.

\section{Workflow-Konzept}
Der technische Workflow wird in folgenden Schritten strukturiert.
Zunächst erfolgt die Skript-Entwicklung durch gemeinsame Erstellung in Google Docs.
Anschließend wird die Gameplay-Aufnahme im \gls{mkv}-Format durchgeführt und auf \gls{nas} hochgeladen.
Die Voice-Over-Aufnahme wird in Audacity mit 48kHz/24Bit \gls{flac} durchgeführt.
Bei der Audio-Bearbeitung werden Noise Reduction und Normalisierung angewendet sowie der Export als \gls{flac} vorgenommen.
Beim Video-Assembly werden alle Assets in DaVinci Resolve importiert.
Der Schnitt basiert auf den Drehbuch-Timings als Rohschnitt.
Bei der Audio-Sync wird die Synchronisation von Voice-Over und Gameplay durchgeführt.
Effekte und Overlays werden durch das Hinzufügen von Text-Overlays und Transitions realisiert.
Falls erforderlich, wird beim Color Grading eine Anpassung der Farbgebung vorgenommen.
Abschließend erfolgt der finale Export in hochauflösendem Format.

\section{Qualitätssicherung}
Zur Sicherstellung einer hohen Produktionsqualität werden folgende Maßnahmen implementiert.
Bezüglich der Audio-Qualität wird vor jeder Aufnahme ein Audio-Test durchgeführt, um Pegel und Rauschunterdrückung zu überprüfen.
Bezüglich der Video-Qualität wird Gameplay-Material auf ausreichende Framerate und Auflösung geprüft.
Die Backup-Strategie sieht vor, dass alle Projektdateien redundant auf dem \gls{nas} und lokal gespeichert werden.
Der Review-Prozess stellt sicher, dass vor dem finalen Export das Video von mehreren Teammitgliedern gesichtet wird.
