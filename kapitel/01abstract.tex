\chapter*{Abstract}\label{abstract}
\addcontentsline{toc}{chapter}{Abstract}
%Deutscher Abstract
Das folgende Dokument soll Informatik-Studierende am FB 2: Duales Studium Wirtschaft · Technik bei der Erstellung ihrer Praxistransferberichte, Studienprojekte und Bachelorarbeit unterstützen. Dafür werden Funktionalitäten von \LaTeX  erklärt und Beispiele gegeben.

%Englischer Abstract
\textit{The following document is intended to support Computer Science students at the DP 2: \textit{Cooperative Studies Business · Technology} in the preparation of their practical transfer reports, study projects and Bachelor's theses. For this purpose, functionalities of \LaTeX  are explained and examples are given.}  % gleicher Text, nur übersetzt

% das gehört NICHT in den PTB, Absatz einfach löschen
%%%%%%%%%%%%%%%%%%%%%%%%%%%%%%%%%%%%%%%%%%%%%%%%%%%%%
\vspace{8em}
\footnotesize Maßgebliche Veränderungen zur vorherigen Version sind: Anpassung der ehrenwörtlichen Erklärung an die derzeitige Studiengangsbeschreibung, Aufteilung der Kapitel in einzelne Dateien, ausführlichere Beispiele zu Abbildungen, Codeausschnitten, Tabellen und der Citation Style entspricht nun APA7.

Um die Vorlage zu verwenden, 
\begin{itemize}
    \vspace{-1em}
    \item ladet ihr die .zip Datei runter,
    \vspace{-1em}
    \item geht in Overleaf auf New Project > Upload Project
    \vspace{-1em}
    \item und wählt dann die .zip aus.
\end{itemize}  
\normalsize
%%%%%%%%%%%%%%%%%%%%%%%%%%%%%%%%%%%%%%%%%%%%%%%%%%%%%

% Bei mehreren Autoren
%%%%%%%%%%%%%%%%%%%%%%%%
%\section*{Schreibverteilung}
%\begin{table}[H]
%\small	
%\begin{tabular}{| l | l |}
%\hline
%Autor   & Kapitel \\ [0.5ex]
%\hline
%\hline
%Vorname & - Kapitel 1\\
%        & - Kapitel 2\\
%        & - Kapitel 3\\
%\hline
%Vorname & - Kapitel 5\\
%        & - Kapitel 6\\
%\hline
%\end{tabular}
%\end{table}
%\clearpage
%%%%%%%%%%%%%%%%%%%%%%%%