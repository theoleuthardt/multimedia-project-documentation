\chapter{Entwurfsentscheidungen}\label{ch:entwurf}
Die gestalterischen und konzeptionellen Entscheidungen für das Video-Review werden auf Basis der definierten Zielgruppe, der zu kommunizierenden Nachricht und medienpsychologischer Prinzipien getroffen.

\section{Visuelle Gestaltung}

\subsection{Szenenaufbau}
Die Szenen des Videos werden wie folgt unterteilt.
Zu Beginn wird eine Intro-Sequenz mit den Reviewern gezeigt, um eine persönliche Verbindung herzustellen und die parasoziale Beziehung zu initiieren.
In erklärenden Segmenten wird Gameplay-Material im Hintergrund gezeigt, während relevante Informationen durch Text-Overlays im Vordergrund präsentiert werden.
Zur klaren Separierung der einzelnen Videoabschnitte werden Übergänge mit Text genutzt, um so das Review klar zu strukturieren.
Zur lustigen Untermalung werden Einblendungen der Reviewer verwendet.

\subsection{\Gls{glos:broll}-Material}
Für bestimmte Segmente wird zusätzliches \gls{glos:broll}-Material eingesetzt.
Bei der Genre-Erklärung werden Ausschnitte aus etablierten Soulslike-Titeln gezeigt, um visuellen Kontext für den Vergleich zu schaffen.
Kritische Momente werden durch spezifisches Gameplay-Material hervorgehoben: Animation Lock bei großen Waffen, Hitbox-Probleme bei kleinen Gegnern, Enemy-Snapping-Effekte, der Target-Lock-Bug und die Schwäche des Endbosses werden gezielt visualisiert.

\subsection{Text-Overlays}
Text-Overlays werden strategisch bei Bugs und kritischen Punkten eingesetzt, um die Aufmerksamkeit zu fokussieren und wichtige Informationen zu verstärken.
Die Overlays werden kurz und prägnant gehalten, um den Lesefluss nicht zu unterbrechen.

\section{Audiogestaltung}

\subsection{Kommentar-Stil}
Der Audio-Kommentar wird im Review-Format aufgenommen, das heißt als durchgehender Voice-Over ohne sichtbare Sprecher während des Hauptteils.
Dies ermöglicht eine kontinuierliche Fokussierung auf das Gameplay-Material.
Der Tonfall wird bewusst energetisch und enthusiastisch gehalten, um die Aufmerksamkeit zu binden.
Pausen werden gezielt für dramatische Effekte eingesetzt.

\subsection{Musikeinsatz}
Es wird eine zweischichtige Musikstrategie verfolgt.
In dynamischen Segmenten wird energetische Musik eingesetzt, um das Pacing zu unterstützen.
In Segmenten, die die Atmosphäre des Spiels vermitteln sollen, wird der \gls{ost} des Spiels verwendet, um authentische Einblicke zu gewähren.

\section{Strukturelle Gestaltung}

\subsection{Informationsarchitektur}
Die Informationen werden in einer logischen Progression präsentiert.
Zunächst wird durch das Intro Aufmerksamkeit geweckt.
Anschließend wird durch die Genre-Erklärung eine Wissensgrundlage geschaffen.
Die visuellen Aspekte vermitteln den ersten Eindruck.
Die Gameplay-Mechaniken stellen die Kernelemente dar.
Die Zusatzfeatures zeigen erweiterte Aspekte.
Die Probleme werden kritisch bewertet.
Abschließend werden im Fazit Zusammenfassung und Empfehlung gegeben.
Diese Struktur folgt dem Prinzip vom Allgemeinen zum Spezifischen und ermöglicht einen natürlichen narrativen Fluss.

\subsection{Pacing-Strategie}
Das Pacing wird durch schnelle Cuts und dynamische Übergänge gesteuert.
Die durchschnittliche Shot-Länge wird bewusst kurz gehalten, um der schnellen Konsumgeschwindigkeit digitaler Medien gerecht zu werden \cite{Rab13}.
Längere, ruhigere Sequenzen werden nur bei der Demonstration kritischer Bugs eingesetzt, um deren Bedeutung zu unterstreichen.

\section{Comedy-Integration}
Die humorvollen Elemente werden subtil in die sachliche Analyse integriert.
Durch überspitzte Formulierungen wird Humor erzeugt, ohne die Informationsvermittlung zu beeinträchtigen.
Das Heranzoomen an die klebenden Charakter-Füße dient als visueller Gag, der gleichzeitig ein echtes Problem illustriert.
Pausen vor Pointen erzeugen komödiantische Effekte.

\section{Zielsetzung der Entwurfsentscheidungen}
Alle gestalterischen Entscheidungen zielen darauf ab, eine Balance zwischen Information und Unterhaltung zu schaffen, die der Zielgruppe gerecht wird.
Es wird angestrebt, ein Review zu produzieren, das sowohl als Kaufentscheidungshilfe dient als auch als unterhaltsamer Content eigenständig funktioniert.
