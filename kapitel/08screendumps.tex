\chapter{Screendumps vom Ergebnis}\label{ch:screendumps}

In diesem Kapitel werden ausgewählte Screendumps des finalen Video-Reviews präsentiert, die zentrale visuelle Aspekte der Produktion dokumentieren.

\section{Intro-Sequenz}

Die Intro-Sequenz stellt den ersten visuellen Kontakt mit dem Zuschauer her und setzt den Ton für das gesamte Review.

% Platzhalter für Screendump der Intro-Sequenz
% \begin{figure}[H]
%     \centering
%     \includegraphics[width=0.9\textwidth]{bilder/screendump_intro.png}
%     \caption{Intro-Sequenz des Video-Reviews (0:00-0:20)}
%     \label{fig:screendump_intro}
% \end{figure}

In diesem Screendump wird der energetische Einstieg visualisiert, bei dem das Spiel \textit{White Lavender} erstmals präsentiert wird. Die visuelle Gestaltung kommuniziert bereits die Comedy-Ausrichtung des Reviews.

\section{Genre-Erklärung mit B-Roll}

Während der theoretischen Erklärung des Soulslike-Genres werden etablierte Titel als Referenz gezeigt.

% Platzhalter für Screendump der Genre-Erklärung
% \begin{figure}[H]
%     \centering
%     \includegraphics[width=0.9\textwidth]{bilder/screendump_genre.png}
%     \caption{B-Roll-Material während der Genre-Erklärung}
%     \label{fig:screendump_genre}
% \end{figure}

Dieser Screendump zeigt die Integration von B-Roll-Material aus bekannten Soulslike-Titeln, die als visueller Kontext für die Erklärung dienen.

\section{Grafikstil-Präsentation}

Ein zentraler Aspekt des Reviews ist die Präsentation des einzigartigen visuellen Stils von \textit{White Lavender}.

% Platzhalter für Screendump des Grafikstils
% \begin{figure}[H]
%     \centering
%     \includegraphics[width=0.9\textwidth]{bilder/screendump_grafik.png}
%     \caption{Präsentation des retro-artigen Grafikstils mit CRT-Filter}
%     \label{fig:screendump_grafik}
% \end{figure}

In diesem Screendump wird der charakteristische retro-artige Stil mit CRT-Filter sichtbar, der das Spiel visuell von anderen Genre-Vertretern abhebt.

\section{Movement-Showcase}

Die Demonstration der Movement-Mechaniken visualisiert eines der diskutierten Gameplay-Elemente.

% Platzhalter für Screendump des Movement-Showcases
% \begin{figure}[H]
%     \centering
%     \includegraphics[width=0.9\textwidth]{bilder/screendump_movement.png}
%     \caption{Movement-Showcase mit Fokus auf Charakter-Animationen}
%     \label{fig:screendump_movement}
% \end{figure}

Dieser Screendump dokumentiert die Präsentation der Bewegungsmechanik, einschließlich der kritisch angemerkten „klebe-am-Boden"-Effekte.

\section{Combat-Szenen}

Die Kampfszenen gegen normale Gegner demonstrieren die Core-Gameplay-Loop.

% Platzhalter für Screendump einer Combat-Szene
% \begin{figure}[H]
%     \centering
%     \includegraphics[width=0.9\textwidth]{bilder/screendump_combat.png}
%     \caption{Combat gegen normale Gegner mit verschiedenen Waffen}
%     \label{fig:screendump_combat}
% \end{figure}

In diesem Screendump werden Kampfsequenzen gezeigt, die sowohl die Waffenvielfalt als auch die diskutierten Probleme wie Hitbox-Issues visualisieren.

\section{Boss-Kämpfe}

Boss-Kämpfe stellen das Herzstück des Soulslike-Genres dar und werden entsprechend prominent präsentiert.

% Platzhalter für Screendump eines Boss-Kampfes
% \begin{figure}[H]
%     \centering
%     \includegraphics[width=0.9\textwidth]{bilder/screendump_boss.png}
%     \caption{Boss-Kampf-Szene mit kreativem Boss-Design}
%     \label{fig:screendump_boss}
% \end{figure}

Dieser Screendump dokumentiert einen der Boss-Kämpfe und illustriert das kreative Design der Gegner.

\section{Endboss-Bug-Demonstration}

Ein kritischer Aspekt des Reviews ist die Demonstration technischer Probleme, insbesondere des Endboss-Bugs.

% Platzhalter für Screendump des Endboss-Bugs
% \begin{figure}[H]
%     \centering
%     \includegraphics[width=0.9\textwidth]{bilder/screendump_endboss_bug.png}
%     \caption{Demonstration des Endboss-Bugs (Charakter steht unter dem Boss)}
%     \label{fig:screendump_endboss_bug}
% \end{figure}

In diesem Screendump wird der diskutierte Bug visualisiert, bei dem der Endboss den Spieler nicht treffen kann, wenn dieser direkt unter ihm steht.

\section{Open World Exploration}

Die Exploration der Spielwelt stellt einen positiv bewerteten Aspekt dar.

% Platzhalter für Screendump der Exploration
% \begin{figure}[H]
%     \centering
%     \includegraphics[width=0.9\textwidth]{bilder/screendump_exploration.png}
%     \caption{Open World Exploration mit Portal-System}
%     \label{fig:screendump_exploration}
% \end{figure}

Dieser Screendump zeigt die verschiedenen Gebiete und das Portal-basierte Reisesystem der Spielwelt.

\section{NPC-Interaktionen}

Die Charakter-Interaktionen werden als positiver Aspekt des Spiels hervorgehoben.

% Platzhalter für Screendump einer NPC-Interaktion
% \begin{figure}[H]
%     \centering
%     \includegraphics[width=0.9\textwidth]{bilder/screendump_npc.png}
%     \caption{NPC-Dialog mit humorvollem Touch}
%     \label{fig:screendump_npc}
% \end{figure}

In diesem Screendump wird ein NPC-Dialog präsentiert, der die charmante Persönlichkeit der Charaktere illustriert.

\section{Fazit-Sequenz}

Die finale Fazit-Sequenz fasst die Bewertung zusammen und präsentiert die Gesamt-Einschätzung.

% Platzhalter für Screendump der Fazit-Sequenz
% \begin{figure}[H]
%     \centering
%     \includegraphics[width=0.9\textwidth]{bilder/screendump_fazit.png}
%     \caption{Fazit-Sequenz}
%     \label{fig:screendump_fazit}
% \end{figure}

In diesem Screendump wird die visuelle Gestaltung des Fazits dokumentiert.