\chapter{Beschreibung des Projektes}\label{ch:projektbeschreibung}

Im Rahmen dieses Multimedia-Projektes wird ein Video-Review zum Indie-Soulslike-Spiel \textit{White Lavender} produziert.
Das Projekt wird als Comedy-Review konzipiert, bei dem eine ehrliche Analyse des Spiels mit humorvollen und unterhaltsamen Elementen kombiniert wird.
Als zentrale Merkmale des Projektes werden sowohl die inhaltliche Bewertung als auch die technische Realisierung eines professionellen Video-Reviews behandelt.
Das Video wird in einer Länge von circa 7:00 bis 8:00 Minuten erstellt und deckt alle relevanten Aspekte eines modernen Game-Reviews ab.

\section{Projektziele und Kontext}

Das Projekt wird im Rahmen des Multimedia-Moduls durchgeführt und verfolgt mehrere Zielsetzungen.
Primär wird praktische Erfahrung in der professionellen Videoproduktion gesammelt, die den gesamten Produktionsprozess von der Konzeption über die Aufnahme bis zur Post-Production umfasst.
Durch die Wahl eines aktuellen Indie-Spiels als Reviewgegenstand wird ein authentischer Use-Case geschaffen, der sich an realen Content-Creator-Workflows orientiert.
Die Entscheidung für das Format eines Comedy-Reviews ermöglicht die Auseinandersetzung mit komplexen narrativen Strukturen, bei denen Informationsvermittlung und Unterhaltung gleichzeitig realisiert werden müssen.
Das Projekt wird in Teamarbeit durchgeführt, wodurch kollaborative Produktionsprozesse und die Koordination verschiedener Produktionsphasen (Skripterstellung, Aufnahme, Schnitt) praktisch erprobt werden.

\section{Erwartetes Ergebnis}

Als finales Ergebnis wird ein veröffentlichungsfähiges Video-Review erstellt, das den Qualitätsstandards professioneller Content-Creator entspricht.
Die Dokumentation des gesamten Produktionsprozesses ermöglicht eine reflektierte Auseinandersetzung mit den getroffenen Entscheidungen und liefert eine Grundlage für zukünftige Multimedia-Projekte.
