%% Vorlage Bachelorarbeit

%% Versionshistorie:

%% v1.0: Erstellung durch Johannes Woske, IT2010, j.woske+latex@gmail.com
%% v2.0: Überarbeitung und Ergänzung durch Anne Traulsen, IT2015, a.traulsen+latex@gmail.com
%% v3.0: Überarbeitung und Ergänzung durch Maja Günther, IT23B, schreibt mir keine Email LG

\documentclass[12pt, a4paper, listof=totoc, bibliography=totoc, numbers=noenddot, ngerman, headsepline, oneside]{scrbook}
\usepackage{amsmath}
\usepackage[T1]{fontenc}
\usepackage{float}
\usepackage[utf8]{inputenc}
\usepackage[ngerman]{babel}
\usepackage{url}
\usepackage{graphicx} 
\usepackage{pdfpages} 
\usepackage{multirow}
\usepackage[a4paper, margin=1in]{geometry}
\usepackage[right]{eurosym} %Euro-Zeichen
\usepackage{amssymb}
\usepackage{subfig}
\usepackage{cite}       % Quellenangaben
\usepackage{setspace}   % Zeilenabstand
\usepackage[ 
   colorlinks,       
   linkcolor=black,   % Farbe interner Verweise 
   filecolor=black,   % Farbe externer Verweise 
   citecolor=black,   % Farbe von Zitaten 
   urlcolor=blue      % Farbe von Links
   ]{hyperref}        %Verlinkungen
\usepackage[figure]{hypcap}
\usepackage[ngerman]{translator}
\usepackage{blindtext} % Lorem-Ipsum-Plugin
\usepackage[acronym, nonumberlist]{glossaries} %% use after hyperref %Glossar-Paket laden
%\usepackage[
%	nonumberlist, %keine Seitenzahlen anzeigen
%	acronym,      %ein Abkürzungsverzeichnis erstellen
%	toc,          %Einträge im Inhaltsverzeichnis
%	section       %im Inhaltsverzeichnis auf section-Ebene erscheinen
%	]
%{glossaries}

\usepackage{listings,xcolor} % Codeanzeige
\usepackage{scrhack}
\usepackage[normalem]{ulem}
\useunder{\uline}{\ul}{}
\usepackage{wrapfig}

\usepackage{makecell}
\usepackage{comment}

% ich würd immer serifenlose Schrift nehmen
% weil wer kann schon Fließtexte in Serifenschrift lesen
% aber es gibt keine offiziellen Vorgaben
\renewcommand{\familydefault}{\sfdefault} % SCHRIFTART, for reference: https://www.overleaf.com/learn/latex/Font_sizes%2C_families%2C_and_styles#Font_families 

\usepackage{pifont}
\usepackage[hashEnumerators,smartEllipses]{markdown}

\usepackage{chngcntr}
\counterwithout{figure}{chapter}
\counterwithout{table}{chapter}

\definecolor{dkgreen}{rgb}{0,.6,0}
\definecolor{dkblue}{rgb}{0.337, 0.612, 0.839}
\definecolor{dkyellow}{rgb}{204, 255, 0}
% hier beginnen lststyle spezifische Farben
\definecolor{lila}{rgb}{0.847, 0.624, 0.855}
\definecolor{hintergrund}{rgb}{0.118, 0.118, 0.118}
\definecolor{blau1}{rgb}{0.573, 0.820, 0.925}
\definecolor{orange}{rgb}{0.839, 0.616, 0.522}
\definecolor{psrot}{rgb}{0.66, 0.18, 0.00}
\definecolor{psblau}{rgb}{0, 0, 0.545}
\definecolor{psbg}{rgb}{0.9176, 0.9490, 0.9804}
\definecolor{psbg2}{rgb}{0.949, 0.949, 0.949}
\definecolor{psbg3}{rgb}{1.0, 0.9843, 0.9608}
\definecolor{key}{rgb}{1.0, 0.573, 0.0}
\definecolor{sep}{rgb}{0.424, 0.188, 0.780}
\definecolor{value}{rgb}{0.231, 0.659, 0.886}

\lstset{
    numbers=left, 
    numberstyle=\tiny\color{black}, 
    numbersep=5pt,
    breaklines=true,
    frame=lr,
    escapeinside={(*@}{@*)}, %nicht anzuzeigende Ausdrücke, z.B. für Labels
    language=[Sharp]C,
    showstringspaces=false,
    captionpos=b,
    backgroundcolor=\color{hintergrund},
    basicstyle=\ttfamily\fontsize{9}{10}\selectfont\color{white},
    keywordstyle    = \color{dkblue},
    stringstyle     = \color{orange},
    identifierstyle = \color{blau1},
    commentstyle    = \color{dkyellow},
    emph            =[1]{var},
    emphstyle       =[1]\color{dkblue},
    emph            =[2]{if,and,or,else, return},
    emphstyle       =[2]\color{lila}
    }    
% mehr Infos zu eigenen Formatierungen: https://de.overleaf.com/learn/latex/Code_listing

\lstdefinestyle{cli}{
    basicstyle=\ttfamily\fontsize{9}{10}\selectfont\color{white},
    identifierstyle = \color{white},
    numbers=none,
    language=sh,
    stringstyle=\color{dkblue},
    emph            =[1]{dotnet},
    emphstyle       =[1]\color{dkyellow},
}
\lstdefinestyle{ps}{
    numbers=left, 
    numberstyle=\tiny\color{black}, 
    numbersep=5pt,
    breaklines=true,
    frame=none,
    escapeinside={(*@}{@*)}, %nicht anzuzeigende Ausdrücke, z.B. für Labels
    language=sh,
    showstringspaces=false,
    captionpos=b,
    backgroundcolor=\color{psbg3},
    basicstyle=\ttfamily\fontsize{9}{10}\selectfont\color{black},
    keywordstyle    = \color{psblau},
    stringstyle     = \color{psrot},
    identifierstyle = \color{psrot},
    commentstyle    = \color{dkyellow},
    emph            =[1]{var},
    emphstyle       =[1]\color{dkblue},
    emph            =[2]{if,and,or,else, return},
    emphstyle       =[2]\color{lila}
}

% wenn Arbeit auf Englisch -> ändern
\renewcommand\lstlistingname{Codeausschnitt}
\renewcommand\lstlistlistingname{Codeverzeichnis}

% Seitenabstände definieren
\geometry{verbose,tmargin=2cm,bmargin=2cm,lmargin=2cm,rmargin=2.3cm} 

\clubpenalty = 10000 \widowpenalty = 10000 \displaywidowpenalty = 10000 

\newcommand{\footfigref}[1]{\footnote{Abb. \ref{#1} auf Seite \pageref{#1}}}

% wenn Arbeit auf Englisch -> zu chapter ändern
\addto\extrasngerman{%
    \def\sectionautorefname{Kapitel}%
    \def\subsectionautorefname{Kapitel}%
    \def\subsubsectionautorefname{Kapitel}%
    }

% Vertikaler Abstand zwischen Ende Textblock - Ende Fußzeile -> Abstand der Seitenzahl von Rand erhöhen 
\setlength{\footskip}{10mm}

\RedeclareSectionCommand[%
    beforeskip=0.5\baselineskip,
    afterskip=0.5\baselineskip
]{chapter}

\RedeclareSectionCommand[%
    beforeskip=0.5\baselineskip,
    afterskip=0.5\baselineskip
]{section}

\RedeclareSectionCommand[%
    beforeskip=0.1\baselineskip,
    afterskip=0.1\baselineskip
]{subsection}

\RedeclareSectionCommand[%
    beforeskip=0.1\baselineskip,
    afterskip=0.1\baselineskip
]{subsubsection}

\RedeclareSectionCommand[%
    beforeskip=0.01\baselineskip,
    %%afterskip=0.2\baselineskip
]{paragraph}

\setlength{\abovecaptionskip}{4pt}  % 1pc=12pt 
\setlength{\belowcaptionskip}{0pt}
%\setlength{\textfloatsep}{4pt}
\setlength{\intextsep}{1pc}

% Verkleinerung der Textgröße unter Abbildungen
\addtokomafont{caption}{\small}

% bei falscher automatischer Silbentrennung wieder einfügen
%\include{hyphenation}

\renewcommand*{\glspostdescription}{}
 
\KOMAoptions{parskip=full*}

% ändert Titelschriftart in Serifen-Normalschriftart
\addtokomafont{disposition}{\rmfamily} 

\makenoidxglossaries

\loadglsentries{glossar.tex}

\newcommand{\type}{Multimedia-Projekt} % oder Bachelorarbeit
\newcommand{\topic}{Projektdokumentation zum Video-Review}
\newcommand{\subtopic}{Ein Comedy-Review zum Indie-Soulslike "White Lavender"}
\newcommand{\studentName}{Theo Leuthardt, Domenik Wilhelm} % mehrere Autoren: {Vorname Name, Vorname Name} etc
\newcommand{\matrikelNr}{72205844868, 77207300494}   % mehrere Autoren: {7220XXXXXXX, 7220XXXXXXX}
\newcommand{\jahrgang}{2025}
\newcommand{\fachbereich}{Duales Studium Wirtschaft · Technik}
\newcommand{\studiengang}{Informatik}
\newcommand{\module}{Multimedia}
\newcommand{\betreuerHS}{(Prof. Dr.) Gerit Kalkbrenner}
\newcommand{\wordCount}{XXXX}


\begin{document}

\author{}
\subject{\type}

\title{
\normalfont\endgraf\rule{\textwidth}{1pt}
\begingroup
	\centering
	\linespread{1.5}
	\huge\topic
\endgroup
\linespread{1.5}
\ \\            % Falls kein Subtopic, auskommentieren
\large\subtopic % Falls kein Subtopic, auskommentieren
\normalfont\endgraf\rule{\textwidth}{1pt}
}

\date{\large vorgelegt am 07. April 2025}
% oder wenn ihr das hier schöner findet:
%%%%%%%%%%%%%%%%%%%%%%%%%%%%%%%%%%%%%%%%
% \date{\normalsize vorgelegt am 7. April 2025\\ \textbullet \\ Fachbereich Duales Studium Wirtschaft · Technik \\
% Hochschule für Wirtschaft und Recht Berlin}
%%%%%%%%%%%%%%%%%%%%%%%%%%%%%%%%%%%%%%%%

\publishers{
	\begin{tabular}{l l}
	\textbf{\normalsize{}} & \normalsize{}  \tabularnewline
	\textbf{\normalsize{}} & \normalsize{}  \tabularnewline
	\textbf{\normalsize{Name:}} & \normalsize{\studentName}  \tabularnewline
 	\textbf{\normalsize{Matrikelnummer:}} & \normalsize{\matrikelNr}  \tabularnewline
	\textbf{\normalsize{Studienjahrgang:}} & \normalsize{\jahrgang}  \tabularnewline
	\textbf{\normalsize{Fachbereich:}} & \normalsize{\fachbereich} \tabularnewline
	\textbf{\normalsize{Studiengang:}} & \normalsize{\studiengang} \tabularnewline
    \textbf{\normalsize{Modul:}} & \normalsize{\module}  \tabularnewline
	\textbf{\normalsize{Betreuer/in Hochschule:}} & \normalsize{\betreuerHS} \tabularnewline
    \textbf{\normalsize{Anzahl der Wörter:}} & \normalsize{\wordCount} \tabularnewline
    \tabularnewline
    \tabularnewline
	\end{tabular}
    % ein Autor:
    %\begin{tabular}{p{15em} p{1em} p{10em}}
    %    \normalsize{Vom Ausbildungsleiter zur Kenntnis genommen:} \tabularnewline
    %    \tabularnewline
    %    \hspace{4cm} && \hspace{4cm}\\\cline{1-1}\cline{3-3}
    %    \normalsize{AL Vorname Name} && \normalsize{DEIN Vorname Name}
    %\end{tabular}
    % mehrere Autoren
    %%%%%%%%%%%%%%%%%
    %\begin{tabular}{p{8em} p{1em} p{7em} p{1em} p{7em}}
    %    \normalsize{Vom Ausbildungsleiter} \\
    %    \normalsize{zur Kenntnis genommen:} \tabularnewline
    %    \hspace{4cm} && \hspace{4cm} && \hspace{4cm}\\\cline{1-1}\cline{3-3}\cline{5-5} 
    %    \normalsize{AL Vorname Name} && \normalsize{DEIN Vorname Name} && \normalsize{DEIN Vorname Name} 
    %\end{tabular}
	%%%%%%%%%%%%%%%%%
    }

\titlehead{\begin{center}
    \includegraphics[height=0.04\textheight]{bilder/HWR.png}
    \hfill
    \end{center}
    }

\maketitle
\onehalfspacing 

% römische Seitenzahlen
\pagenumbering{Roman}

% Abstract (ausgelagert in extra File)
% \chapter*{Abstract}\label{abstract}
\addcontentsline{toc}{chapter}{Abstract}
%Deutscher Abstract
Das folgende Dokument soll Informatik-Studierende am FB 2: Duales Studium Wirtschaft · Technik bei der Erstellung ihrer Praxistransferberichte, Studienprojekte und Bachelorarbeit unterstützen. Dafür werden Funktionalitäten von \LaTeX  erklärt und Beispiele gegeben.

%Englischer Abstract
\textit{The following document is intended to support Computer Science students at the DP 2: \textit{Cooperative Studies Business · Technology} in the preparation of their practical transfer reports, study projects and Bachelor's theses. For this purpose, functionalities of \LaTeX  are explained and examples are given.}  % gleicher Text, nur übersetzt

% das gehört NICHT in den PTB, Absatz einfach löschen
%%%%%%%%%%%%%%%%%%%%%%%%%%%%%%%%%%%%%%%%%%%%%%%%%%%%%
\vspace{8em}
\footnotesize Maßgebliche Veränderungen zur vorherigen Version sind: Anpassung der ehrenwörtlichen Erklärung an die derzeitige Studiengangsbeschreibung, Aufteilung der Kapitel in einzelne Dateien, ausführlichere Beispiele zu Abbildungen, Codeausschnitten, Tabellen und der Citation Style entspricht nun APA7.

Um die Vorlage zu verwenden, 
\begin{itemize}
    \vspace{-1em}
    \item ladet ihr die .zip Datei runter,
    \vspace{-1em}
    \item geht in Overleaf auf New Project > Upload Project
    \vspace{-1em}
    \item und wählt dann die .zip aus.
\end{itemize}  
\normalsize
%%%%%%%%%%%%%%%%%%%%%%%%%%%%%%%%%%%%%%%%%%%%%%%%%%%%%

% Bei mehreren Autoren
%%%%%%%%%%%%%%%%%%%%%%%%
%\section*{Schreibverteilung}
%\begin{table}[H]
%\small	
%\begin{tabular}{| l | l |}
%\hline
%Autor   & Kapitel \\ [0.5ex]
%\hline
%\hline
%Vorname & - Kapitel 1\\
%        & - Kapitel 2\\
%        & - Kapitel 3\\
%\hline
%Vorname & - Kapitel 5\\
%        & - Kapitel 6\\
%\hline
%\end{tabular}
%\end{table}
%\clearpage
%%%%%%%%%%%%%%%%%%%%%%%%
% \newpage

% Inhaltsverzeichnis
\tableofcontents{}
% ändern, wenn englische Arbeit
\addcontentsline{toc}{chapter}{Inhaltsverzeichnis}
\clearpage

% ich würd die folgenden drei Verzeichnisse immer nachs KI Verzeichnis, vor den Anhang einfügen
% damit ihr nicht noch voem Text drei halbleere Seiten habt
% aber Frau Monett Diaz meinte, nach dem Inhaltsverzeichnis ist besser (:

% Abbildungsverzeichnis -> auskommentieren, wenn keine Abbildungen verwendet wurden
\listoffigures
\clearpage

% Tabellenverzeichnis -> auskommentieren, wenn keine Tabellen verwendet wurden
% \listoftables
% \clearpage

% Codeverzeichnis -> auskommentieren, wenn keine Codeausschnitte verwendet wurden
% \lstlistoflistings
% \clearpage

% Akronyme extra
\addcontentsline{toc}{chapter}{Akronyme}
\printnoidxglossary[type=\acronymtype]
\clearpage

% Glossar
\addcontentsline{toc}{chapter}{Glossar}
\printnoidxglossary
\clearpage

% arabische Seitenzahlen sobalds losgeht
\pagenumbering{arabic}

% alle Kapitel schick ausgelagert
\chapter{Beschreibung des Projektes}\label{ch:projektbeschreibung}

Im Rahmen dieses Multimedia-Projektes wird ein Video-Review zum Indie-Soulslike-Spiel \textit{White Lavender} produziert.
Das Projekt wird als Comedy-Review konzipiert, bei dem eine ehrliche Analyse des Spiels mit humorvollen und unterhaltsamen Elementen kombiniert wird.
Als zentrale Merkmale des Projektes werden sowohl die inhaltliche Bewertung als auch die technische Realisierung eines professionellen Video-Reviews behandelt.
Das Video wird in einer Länge von circa 7:00 bis 8:00 Minuten erstellt und deckt alle relevanten Aspekte eines modernen Game-Reviews ab.

\section{Projektziele und Kontext}

Das Projekt wird im Rahmen des Multimedia-Moduls durchgeführt und verfolgt mehrere Zielsetzungen.
Primär wird praktische Erfahrung in der professionellen Videoproduktion gesammelt, die den gesamten Produktionsprozess von der Konzeption über die Aufnahme bis zur Post-Production umfasst.
Durch die Wahl eines aktuellen Indie-Spiels als Reviewgegenstand wird ein authentischer Use-Case geschaffen, der sich an realen Content-Creator-Workflows orientiert.
Die Entscheidung für das Format eines Comedy-Reviews ermöglicht die Auseinandersetzung mit komplexen narrativen Strukturen, bei denen Informationsvermittlung und Unterhaltung gleichzeitig realisiert werden müssen.
Das Projekt wird in Teamarbeit durchgeführt, wodurch kollaborative Produktionsprozesse und die Koordination verschiedener Produktionsphasen (Skripterstellung, Aufnahme, Schnitt) praktisch erprobt werden.

\section{Erwartetes Ergebnis}

Als finales Ergebnis wird ein veröffentlichungsfähiges Video-Review erstellt, das den Qualitätsstandards professioneller Content-Creator entspricht.
Die Dokumentation des gesamten Produktionsprozesses ermöglicht eine reflektierte Auseinandersetzung mit den getroffenen Entscheidungen und liefert eine Grundlage für zukünftige Multimedia-Projekte.

\chapter{Zielgruppe}\label{ch:zielgruppe}
Die Zielgruppe des Video-Reviews wird primär aus aktiven Gamern und Streamern gebildet.
Diese Personengruppe zeichnet sich durch ein ausgeprägtes Interesse an Videospielen, insbesondere im Bereich der Action-\glspl{rpg} und Soulslike-Titel, aus.

\section{Primäre Zielgruppe}
Als Kernzielgruppe werden Personen definiert, die regelmäßig Videospiele konsumieren und aktiv nach Informationen über neue oder weniger bekannte Indie-Titel suchen.
Diese Gruppe wird in folgende Segmente unterteilt: Soulslike-Enthusiasten sind Spieler, die bereits Erfahrungen mit bekannten Genre-Vertretern wie \textit{Dark Souls} oder \textit{Elden Ring} gesammelt haben und an alternativen Interpretationen des Genres interessiert sind.
Indie-Game-Interessierte sind Personen, die gezielt nach ungewöhnlichen und experimentellen Spielen außerhalb des \gls{aaa}-Mainstreams suchen und Wert auf einzigartige Artstyles und innovative Konzepte legen.
Content Creator und Streamer sind Gaming-Streamer und YouTuber, die nach interessantem Content für ihre Kanäle suchen und besonders an Spielen mit hohem Unterhaltungswert und Diskussionspotenzial interessiert sind.

\section{Sekundäre Zielgruppe}
Als sekundäre Zielgruppe werden Personen identifiziert, die zwar nicht zur Kernzielgruppe gehören, aber dennoch durch das Format angesprochen werden können.
Casual Gamer sind Spieler, die nur gelegentlich spielen, aber durch den humorvollen Ansatz des Reviews und die unterhaltsame Präsentation angezogen werden.
Retro-Gaming-Fans sind Personen, die sich für den retro-artigen Grafikstil und den nostalgischen \gls{crt}-Filter-Look interessieren.

\section{Zielgruppengerechte Ansprache}
Die Ansprache der Zielgruppe erfolgt durch eine Kombination aus fachlich fundierter Analyse und unterhaltsamer Comedy-Präsentation.
Es wird davon ausgegangen, dass die Zuschauer bereits über Grundkenntnisse in Gaming-Begrifflichkeiten verfügen, jedoch werden komplexere Konzepte wie das Soulslike-Genre explizit erklärt.
Der Tonfall wird bewusst locker und authentisch gehalten, um die Community-Nähe zu wahren, die für Gaming-Content charakteristisch ist.

\chapter{Zu kommunizierende Nachricht}\label{ch:nachricht}
Die zentrale Botschaft des Video-Reviews wird auf mehreren Ebenen vermittelt.
Es wird sowohl eine Kritikäußerung am Spiel, als auch eine grundlegende Wissensvermittlung über das Soulslike-Genre angestrebt.

\section{Primäre Botschaft}
Als Hauptbotschaft wird vermittelt, dass \textit{White Lavender} ein ambitioniertes Indie-Soulslike mit großem Potential darstellt, das jedoch zum aktuellen Zeitpunkt noch unter erheblichen technischen Problemen und Design-Schwächen leidet.
Es wird eine differenzierte Bewertung kommuniziert, die sowohl die positiven als auch die negativen Aspekte des Spiels transparent darstellt.
Die Kernaussage lautet: Das Spiel bietet einen erfrischend anderen Ansatz im Soulslike-Genre durch seinen einzigartigen visuellen Stil und die bunte Welt, kann aber aufgrund von Bugs, unausgewogener Schwierigkeit und technischen Mängeln aktuell nur bedingt empfohlen werden.

\section{Sekundäre Botschaften}
Neben der Hauptbotschaft werden folgende Teilaspekte kommuniziert.
Es wird grundlegendes Wissen über Soulslike-Spiele vermittelt, indem die charakteristischen Merkmale des Genres erklärt werden.
Dadurch werden auch Zuschauer ohne Soulslike-Erfahrung in die Lage versetzt, die Bewertung nachzuvollziehen.
Es wird kommuniziert, dass technische Probleme bei Indie-Titeln nicht ungewöhnlich sind und dass das kreative Potential des Spiels anerkannt wird.
Die Kritik wird konstruktiv formuliert, ohne die Leistung der Entwickler zu diskreditieren.
Es wird vermittelt, dass eine ehrliche und transparente Bewertung wichtiger ist als blindes Lob.
Die identifizierten Probleme werden klar benannt.
Es wird eine klare Empfehlung ausgesprochen: Das Spiel richtet sich primär an hardcore Indie-Fans, die über technische Probleme hinwegsehen können.
Für die breite Masse wird empfohlen, weitere Entwicklungszyklen abzuwarten.

\section{Tonalität der Nachricht}
Die Nachricht wird in einem humorvollen, aber respektvollen Ton vermittelt.
Es wird darauf geachtet, dass die Comedy-Elemente die sachliche Kritik nicht untergraben, sondern die Botschaft zugänglicher und unterhaltsamer machen.
Die Balance zwischen Unterhaltung und Information wird als wesentliches Element der Kommunikationsstrategie verstanden.

\chapter{Medienpsychologische Begründung}\label{ch:medienpsychologie}
Die Konzeption des Video-Reviews basiert auf verschiedenen medienpsychologischen Prinzipien, die eine effektive Informationsvermittlung und hohe Zuschauerbindung gewährleisten sollen.

\section{Aufmerksamkeitsbindung durch Strukturierung}
Das Video wird in klar abgegrenzte Segmente unterteilt, die jeweils einen spezifischen Aspekt des Spiels behandeln.
Diese Strukturierung folgt dem Prinzip der kognitiven Entlastung: Durch die Aufteilung in überschaubare Informationseinheiten wird die Informationsverarbeitung erleichtert und die Aufmerksamkeit über die gesamte Laufzeit aufrechterhalten \cite{Lan00}.
Die durchschnittliche Segmentlänge von 20-30 Sekunden entspricht der optimalen Aufmerksamkeitsspanne für digitale Videoinhalte und verhindert kognitive Überlastung \cite{Cut16}.

\section{Duale Kodierung}
Es wird das Prinzip der dualen Kodierung nach Paivio angewendet: Informationen werden sowohl visuell (durch Gameplay-Material, \gls{glos:broll}, Text-Overlays) als auch auditiv (durch gesprochenen Kommentar) präsentiert \cite{Pai90}.
Diese multimodale Präsentation führt zu einer verbesserten Enkodierung im Gedächtnis und erhöht die Behaltensleistung der vermittelten Informationen \cite{May02}.
Kritische Inhalte wie Bugs und technische Probleme werden zusätzlich durch Text-Overlays hervorgehoben, um eine dreifache Kodierung (visuell-bildlich, visuell-textlich, auditiv) zu erreichen.

\section{Emotionale Aktivierung}
Durch die Integration humorvoller Elemente wird eine positive emotionale Aktivierung erzeugt.
Diese dient mehreren Zwecken: Emotional aktivierende Inhalte werden intensiver verarbeitet und besser erinnert als neutral präsentierte Informationen.
Negative Kritikpunkte werden durch humorvolle Präsentation entschärft und somit eher akzeptiert.
Die Zielgruppe zeigt eine geringere Abwehrhaltung gegenüber kritischen Bewertungen, wenn diese unterhaltsam verpackt werden.
Der lockere, authentische Tonfall fördert die parasoziale Beziehung zwischen Rezipient und Content Creator, was zu höherer Glaubwürdigkeit und Vertrauen führt \cite{Hor56}.

\section{Pacing und dynamische Gestaltung}
Das Video wird mit schnellen Cuts und dynamischen Übergängen gestaltet, um dem Konsumverhalten der digitalen Zielgruppe gerecht zu werden.
Diese Gestaltungsprinzipien entsprechen den Sehgewohnheiten von Zuschauern, die an schnelle Content-Formate gewöhnt sind \cite{Cut16}.
Die energetische Intro-Sequenz aktiviert das Aufmerksamkeitssystem und signalisiert dem Zuschauer unmittelbar die Tonalität und den Unterhaltungswert des Videos.

\section{Konkrete Visualisierung}
Kritische Spielaspekte werden durch konkrete Gameplay-Beispiele visualisiert.
Diese Konkretisierung folgt dem Prinzip des \textit{Showing, not telling}: Anstatt Probleme nur zu beschreiben, werden sie direkt gezeigt.
Dies erhöht die Glaubwürdigkeit der Aussagen und ermöglicht es den Zuschauern, eigene Urteile zu bilden.

\section{Wiederholung und Verstärkung}
Zentrale Botschaften werden mehrfach im Video adressiert und im Fazit zusammengefasst.
Diese Wiederholung dient der Verfestigung der Kerninformationen im Langzeitgedächtnis.

\section{Balancierte Argumentation}
Es wird sowohl auf positive als auch auf negative Aspekte eingegangen.
Diese Form der Argumentation wird von informierten Zielgruppen als glaubwürdiger wahrgenommen als einseitige Darstellungen und erhöht die Überzeugungskraft der finalen Bewertung \cite{All09}.

\chapter{Entwurfsentscheidungen}\label{ch:entwurf}
Die gestalterischen und konzeptionellen Entscheidungen für das Video-Review werden auf Basis der definierten Zielgruppe, der zu kommunizierenden Nachricht und medienpsychologischer Prinzipien getroffen.

\section{Visuelle Gestaltung}

\subsection{Szenenaufbau}
Die Szenen des Videos werden wie folgt unterteilt.
Zu Beginn wird eine Intro-Sequenz mit den Reviewern gezeigt, um eine persönliche Verbindung herzustellen und die parasoziale Beziehung zu initiieren.
In erklärenden Segmenten wird Gameplay-Material im Hintergrund gezeigt, während relevante Informationen durch Text-Overlays im Vordergrund präsentiert werden.
Zur klaren Separierung der einzelnen Videoabschnitte werden Übergänge mit Text genutzt, um so das Review klar zu strukturieren.
Zur lustigen Untermalung werden Einblendungen der Reviewer verwendet.

\subsection{\Gls{glos:broll}-Material}
Für bestimmte Segmente wird zusätzliches \gls{glos:broll}-Material eingesetzt.
Bei der Genre-Erklärung werden Ausschnitte aus etablierten Soulslike-Titeln gezeigt, um visuellen Kontext für den Vergleich zu schaffen.
Kritische Momente werden durch spezifisches Gameplay-Material hervorgehoben: Animation Lock bei großen Waffen, Hitbox-Probleme bei kleinen Gegnern, Enemy-Snapping-Effekte, der Target-Lock-Bug und die Schwäche des Endbosses werden gezielt visualisiert.

\subsection{Text-Overlays}
Text-Overlays werden strategisch bei Bugs und kritischen Punkten eingesetzt, um die Aufmerksamkeit zu fokussieren und wichtige Informationen zu verstärken.
Die Overlays werden kurz und prägnant gehalten, um den Lesefluss nicht zu unterbrechen.

\section{Audiogestaltung}

\subsection{Kommentar-Stil}
Der Audio-Kommentar wird im Review-Format aufgenommen, das heißt als durchgehender Voice-Over ohne sichtbare Sprecher während des Hauptteils.
Dies ermöglicht eine kontinuierliche Fokussierung auf das Gameplay-Material.
Der Tonfall wird bewusst energetisch und enthusiastisch gehalten, um die Aufmerksamkeit zu binden.
Pausen werden gezielt für dramatische Effekte eingesetzt.

\subsection{Musikeinsatz}
Es wird eine zweischichtige Musikstrategie verfolgt.
In dynamischen Segmenten wird energetische Musik eingesetzt, um das Pacing zu unterstützen.
In Segmenten, die die Atmosphäre des Spiels vermitteln sollen, wird der \gls{ost} des Spiels verwendet, um authentische Einblicke zu gewähren.

\section{Strukturelle Gestaltung}

\subsection{Informationsarchitektur}
Die Informationen werden in einer logischen Progression präsentiert.
Zunächst wird durch das Intro Aufmerksamkeit geweckt.
Anschließend wird durch die Genre-Erklärung eine Wissensgrundlage geschaffen.
Die visuellen Aspekte vermitteln den ersten Eindruck.
Die Gameplay-Mechaniken stellen die Kernelemente dar.
Die Zusatzfeatures zeigen erweiterte Aspekte.
Die Probleme werden kritisch bewertet.
Abschließend werden im Fazit Zusammenfassung und Empfehlung gegeben.
Diese Struktur folgt dem Prinzip vom Allgemeinen zum Spezifischen und ermöglicht einen natürlichen narrativen Fluss.

\subsection{Pacing-Strategie}
Das Pacing wird durch schnelle Cuts und dynamische Übergänge gesteuert.
Die durchschnittliche Shot-Länge wird bewusst kurz gehalten, um der schnellen Konsumgeschwindigkeit digitaler Medien gerecht zu werden \cite{Rab13}.
Längere, ruhigere Sequenzen werden nur bei der Demonstration kritischer Bugs eingesetzt, um deren Bedeutung zu unterstreichen.

\section{Comedy-Integration}
Die humorvollen Elemente werden subtil in die sachliche Analyse integriert.
Durch überspitzte Formulierungen wird Humor erzeugt, ohne die Informationsvermittlung zu beeinträchtigen.
Das Heranzoomen an die klebenden Charakter-Füße dient als visueller Gag, der gleichzeitig ein echtes Problem illustriert.
Pausen vor Pointen erzeugen komödiantische Effekte.

\section{Zielsetzung der Entwurfsentscheidungen}
Alle gestalterischen Entscheidungen zielen darauf ab, eine Balance zwischen Information und Unterhaltung zu schaffen, die der Zielgruppe gerecht wird.
Es wird angestrebt, ein Review zu produzieren, das sowohl als Kaufentscheidungshilfe dient als auch als unterhaltsamer Content eigenständig funktioniert.

\chapter{Drehbuch}\label{ch:drehbuch}

Das Drehbuch wird in elf klar definierte Szenen unterteilt, die zusammen eine Gesamtlaufzeit von circa 7:00 bis 8:00 Minuten ergeben. Jede Szene wird hinsichtlich Inhalt, visueller Umsetzung und Audio-Kommentar detailliert spezifiziert.

\section{Szenenübersicht}

\subsection{Szene 1: Intro (0:00-0:20)}

\textbf{Titel:} INTRO

\textbf{Inhalt:} Ein energetischer Einstieg ins Review wird geschaffen, bei dem das Spiel und seine Besonderheiten vorgestellt werden.

\textbf{Visuelle Umsetzung:} Als energetischer Einstieg wird eine dynamische Intro-Sequenz gezeigt.

\textbf{Audio-Kommentar:} „Was passiert, wenn ihr Soulslikes nehmt, einen CRT-Filter drüberkippt und das Ganze aussehen lasst wie ein Fiebertraum aus den 90ern? Richtig – White Lavender! Heute schauen wir uns dieses bunte Indie-Soulslike an, das beweist, dass man nicht immer düster und ernst sein muss, um schwer zu sein. Spoiler: Es ist chaotisch, es ist bunt, und ja – es hat Bugs. Aber lohnt es sich trotzdem? Let's go!"

\subsection{Szene 2: Soulslikes (0:20-0:45)}

\textbf{Titel:} SOULSLIKES

\textbf{Inhalt:} Eine theoretische Erklärung mit Inhalten, was Soulslikes ausmacht, wird präsentiert.

\textbf{Visuelle Umsetzung:} Als \gls{glos:broll} werden Ausschnitte aus \textit{Dark Souls}, \textit{Elden Ring} und anderen Genre-Vertretern gezeigt.

\textbf{Audio-Kommentar:} „Für die, die noch nie ein Soulslike gespielt haben – hier ein Crash-Course: Soulslikes sind Action-\glspl{rpg}, die für ihre knallharte Schwierigkeit berüchtigt sind. Ihr seid ein Kämpfer in einer Welt voller Monster und Gefahren – das Ziel? Überleben und stärker werden. Präzises Timing, Ausdauer-Management, Bosskämpfe die euch an den Rand des Wahnsinns treiben, und wenn ihr sterbt? Dann dürft ihr schön zurücklaufen und eure Seelen einsammeln... falls ihr es bis dahin schafft. Die bekanntesten Vertreter? Dark Souls, Elden Ring, Bloodborne – Spiele die legendär dafür sind, dass Gamer ihre Controller an die Wand werfen. White Lavender nimmt diese Formel und dreht den Weird-Regler auf Maximum."

\subsection{Szene 3: Grafik (0:45-1:15)}

\textbf{Titel:} GRAFIK

\textbf{Inhalt:} Der super einzigartige retro-artige Grafikstil wird präsentiert: sehr bunt, mit \gls{crt}-Filter, lustigem Charakterdesign und sichtbarer Rüstung am Charakter.

\textbf{Visuelle Umsetzung:} Gameplay von \textit{White Lavender} wird mit Fokus auf die Visuals gezeigt. An die Charakter-Füße wird herangezoomt.

\textbf{Audio-Kommentar:} „Optisch ist das Spiel einfach nicht so das, was man vom Genre kennt. Aber im guten Sinne! Der retro-artige Grafikstil mit dem \gls{crt}-Filter sieht aus, als würde man auf einem alten Röhrenfernseher zocken. Alles ist super bunt und poppig – das komplette Gegenteil von den düsteren Souls-Spielen, die am bekanntesten in dem Genre sind. Das Charakterdesign ist absolut goofy und liebenswert. Und das Beste: Eure Rüstung ist tatsächlich am Charakter sichtbar! Fashion Souls könnte man schon fast sagen. Aber schaut euch mal diese Füße an... die kleben förmlich am Boden. Das Movement ist ebenfalls... nun ja, auch echt speziell. Aber dazu kommen wir jetzt."

\subsection{Szene 4: Movement (1:15-1:35)}

\textbf{Titel:} MOVEMENT

\textbf{Inhalt:} Das Movement-System des Spiels wird vorgestellt.

\textbf{Visuelle Umsetzung:} Ein Movement-Showcase wird präsentiert, bei dem Movement und Fähigkeiten am Anfang des Spiels gezeigt werden.

\textbf{Audio-Kommentar:} „Also, sprechen wir über Movement. Am Anfang fühlt sich euer Charakter an wie... naja, wie jemand der in Honig watet. Ihr habt die Basic-Bewegungen: Laufen, Rollen, Springen – aber es fühlt sich etwas schwerfällig an. Die Füße haben wirklich diesen 'klebe-am-Boden-Effekt', was dem Ganzen einen ungewollt komischen Touch gibt. Aber hey, sobald ihr ein paar Fähigkeiten freischaltet und euch an die Physik gewöhnt habt, wird's besser. Es ist anders, aber man gewöhnt sich dran."

\subsection{Szene 5: Enemies (1:35-2:00)}

\textbf{Titel:} ENEMIES

\textbf{Inhalt:} Der Combat gegen normale Gegner wird im Detail bewertet.

\textbf{Visuelle Umsetzung:} Kampfszenen gegen normale Gegner werden gezeigt, Hitbox-Probleme werden demonstriert und verschiedene Waffen werden präsentiert.

\textbf{Audio-Kommentar:} „Der Combat gegen normale Gegner macht grundsätzlich Spaß – aber er hat leider seine Macken. Timing-basiertes Angreifen, Ausweichen, Ausdauer-Management – das Soulslike-Package ist vollstens vorhanden. ABER: Gerade bei größeren Waffen merkt man den Animation Lock heftig. Einmal geschwungen, seid ihr committed – und dann kanns tödlich sein. Und dann die Hitboxen... besonders bei kleineren Gegnern gehen eure Angriffe manchmal einfach daneben. Das ist frustrierend, wenn ihr wisst, dass der Hit eigentlich sitzen sollte. Aber positiv ist: Das Spiel fördert echt das Waffenexperimentieren! Verschiedene Gebiete, verschiedene Gegnertypen – da lohnt es sich, mehrere Waffen auszuprobieren. Die Waffenvielfalt ist cool, von Standard bis komplett abgedreht."

\subsection{Szene 6: Bosse (2:00-2:30)}

\textbf{Titel:} BOSSE

\textbf{Inhalt:} Das Bossdesign und die Schwierigkeit des Spiels werden bewertet.

\textbf{Visuelle Umsetzung:} Boss-Kampf-Footage wird präsentiert, einschließlich des Endboss-Kampfes und Kampfszenen gegen verschiedene Bosse.

\textbf{Audio-Kommentar:} „Und dann kommen wir zum Herzstück eines jeden Soulslikes, den Bossen... ein zweischneidiges Schwert in diesem Spiel. Das Bossdesign ist kreativ und jeder Boss sieht unique aus. ABER: Die Patterns sind manchmal echt simpel und langweilig. Nach ein, zwei Versuchen habt ihr die meisten Bosse durchschaut. Der Endboss ist im Vergleich zu den anderen Bossen extrem schwach. Wir haben sogar diesen Bug gefunden: [PAUSE] Wenn man direkt unter ihm steht, kann er euch nicht treffen. Da steht man dann und wundert sich... [PAUSE] Die Schwierigkeit ist also sehr inkonsistent – manche Bosse machen euch das Leben zur Hölle und andere macht ihr im Schlaf."

\subsection{Szene 7: Erkunden (2:25-2:45)}

\textbf{Titel:} ERKUNDEN

\textbf{Inhalt:} Das Erlebnis beim Erkunden der Open World wird bewertet.

\textbf{Visuelle Umsetzung:} Open World Exploration wird gezeigt, Portal-Reisen werden demonstriert, das Aufheben von Items wird präsentiert, und verschiedene Gebiete werden durch Portale bereist. Verschiedene Items, die über die Map auffindbar sind, werden gezeigt.

\textbf{Audio-Kommentar:} „Die Open World zu erkunden macht echt Laune! Ihr reist durch verschiedene Gebiete via Portale – und jedes Gebiet hat seinen eigenen Vibe. Von bunten Wäldern, über dunkle Höhlen bis zu trippy Landschaften ist alles dabei. Überall findet ihr Items, Geheimnisse und versteckte Wege. Es lohnt sich also definitiv, jeden Winkel zu erkunden. [PAUSE] Aber das Gefühl, wenn man einen versteckten Schatz findet? \textit{Chef's kiss} Unbezahlbar"

\subsection{Szene 8: \glspl{npc} (2:45-3:05)}

\textbf{Titel:} \glspl{npc}

\textbf{Inhalt:} Die \gls{npc}-Interaktionen werden bewertet hinsichtlich deren Qualität, Tiefe, Questlines und humorvollem Touch.

\textbf{Visuelle Umsetzung:} \gls{npc}-Gespräche und Szenen beim Reden mit \glspl{npc} werden gezeigt.

\textbf{Audio-Kommentar:} „Die \glspl{npc} in White Lavender sind sogar überraschend charmant! Die Dialoge haben einen humorvollen Touch und die Questlines sind echt interessant. Klar, es ist jetzt keine narrative Meisterleistung wie man es von \gls{aaa}-Titeln kennt, aber die \glspl{npc} haben Persönlichkeit und sorgen für ein paar Lacher während des Gameplays. Insgesamt fühlen sich die Interaktionen lebendiger an als man von einem Indie-Titel vielleicht erwarten würde."

\subsection{Szene 9: Musik (3:05-3:20)}

\textbf{Titel:} MUSIK

\textbf{Inhalt:} Es wird kurz erklärt, was für Musik genutzt wird (Genre) und ob es zum Charakter des Spiels passt.

\textbf{Visuelle Umsetzung:} Hintergrund-Gameplay mit Musik wird präsentiert, Hintergrundgameplay von \textit{White Lavender} läuft.

\textbf{Audio-Kommentar:} „Der Soundtrack passt perfekt zum psychedelischen Vibe des Spiels. Es ist oft diese Mischung aus ambient und retro-synth Sounds, die das Ganze zusammenhält. Zusätzlich sind auch wirklich häufig Banger dabei, die man jetzt nicht so erwartet hat. Die Musik untermalt so die sowohl entspannenden Momente wie das Erkunden in der Welt als auch die sehr aufregenden Momente des Spiel wie die Bosskämpfe."

\subsection{Szene 10: Bugs (3:20-3:45)}

\textbf{Titel:} BUGS

\textbf{Inhalt:} Aktuelle Probleme des Spiels und Fehler beim eigenen Durchlauf werden präsentiert.

\textbf{Visuelle Umsetzung:} Bug-Footage wird gezeigt, spezifische Bug-Beispiele werden demonstriert, der Endboss-Bug wird nochmal gezeigt, spezifische Ausschnitte der Bugs werden präsentiert.

\textbf{Audio-Kommentar:} „Okay, jetzt müssen wir noch über den Elefanten im Raum sprechen: nämlich die Bugs im Spiel. Und oh junge, die Liste ist lang. Wir hatten: Enemy Snapping-Probleme – Gegner teleportieren sich plötzlich, Clipping-Issues und Gegner die in Wänden stecken, Framerate-Drops, und der absolute Kracher: Am Ende unseres Runs ließen sich Gegner GAR NICHT MEHR anvisieren. Also so komplett. Das Target-Lock-System hat einfach aufgegeben. Und wie gesagt, der Endboss kann euch nicht treffen wenn ihr unter ihm steht. Das ist jetzt weniger ein Bug und mehr ein 'haben wir nicht gebalanced'-Problem. Für ein Indie-Spiel ist das nicht ungewöhnlich, aber aktuell müsst ihr mit diesen technischen Problemen leider leben. Die Devs patchen hoffentlich fleißig, [PAUSE, dann leise] Gott bewahre..."

\subsection{Szene 11: Fazit (3:45-4:10)}

\textbf{Titel:} FAZIT

\textbf{Inhalt:} Ein Scoring System und eine Bewertung werden über das Video zusammengefasst.

\textbf{Visuelle Umsetzung:} Eine Zusammenfassung mit Best-of Footage wird präsentiert. Im Hintergrund läuft unscharfes Gameplay. Im Vordergrund werden Grafiken der vergebenen Scorings pro Videosektion gezeigt. Ein Outro wird eingeblendet.

\textbf{Audio-Kommentar:} „White Lavender ist ein ambitioniertes Indie-Soulslike mit einer richtig coolen Idee. Der einzigartige Artstyle und die bunte Welt sind erfrischend anders. Das Waffenexperimentieren macht Spaß und die notwendige Varietät ist da. ABER: Die technischen Probleme und Design-Schwächen ziehen das Erlebnis deutlich runter. Animation Lock, miese Hitboxen, simple Bosspatterns, ein viel zu schwacher Endboss, und Bugs die das Spiel teilweise unspielbar machen – das muss man als Spieler auch erstmal schlucken, wenn man das zu spät nach einem Kauf merkt. Es fühlt sich an wie ein Early Access-Titel der noch ein paar Monate Entwicklung braucht. Das Potential ist da, aber aktuell ist es eher was für die hardcore Indie-Fans, die über technische Probleme hinwegsehen können. Wenn die Devs weiter dran arbeiten sollten, könnte das wirklich eine gute Abwechslung zu den eher bekannten Soulslike-Titeln wie eben Dark Souls oder Elden Ring werden. Aber im aktuellen Zustand? Eher schwierig. Habt ihr White Lavender schon gespielt und was sind eure Erfahrungen mit Soulslike Spielen? Lasst es und gerne wissen und bis zum nächsten Mal!"

\chapter{Technische Entscheidungen}\label{ch:technik}
Die technische Realisierung des Video-Reviews basiert auf professionellen Werkzeugen und Formaten, die eine hohe Qualität und effiziente Zusammenarbeit gewährleisten.

\section{Videobearbeitung}

\subsection{Editor-Wahl}
Als primäres Bearbeitungswerkzeug wird DaVinci Resolve eingesetzt.
Diese Wahl wird durch folgende Faktoren begründet: DaVinci Resolve bietet umfassende Bearbeitungsfunktionen für Schnitt, Color Grading und Effekte, die für ein professionelles Review erforderlich sind.
Die kostenlose Version von DaVinci Resolve bietet bereits einen umfangreichen Funktionsumfang, der für dieses Projekt ausreichend ist.
Das Programm ermöglicht einen effizienten Workflow von der Rohschnitt-Phase über das Color Grading bis zur finalen Ausgabe.

\subsection{Videoformat}
Für die Verarbeitung werden die Containerformate \gls{mkv} und \gls{mov} verwendet.
\gls{mkv} wird bevorzugt für Gameplay-Aufnahmen, Zwischenspeicherungen während der Bearbeitung, hohe Flexibilität bei Codec-Wahl und verlustfreie Qualität.
\gls{mov} wird verwendet für Kompatibilität mit DaVinci Resolve, professionelle Farbbearbeitung und standardisierte Ausgabeformate.

\section{Audiobearbeitung}

\subsection{Aufnahme-Software}
Für die Audioaufnahme wird Audacity eingesetzt.
Diese Open-Source-Software wird aufgrund folgender Eigenschaften gewählt: Audacity ist kostenlos verfügbar und bietet dennoch professionelle Aufnahmemöglichkeiten.
Die Benutzeroberfläche ist intuitiv und ermöglicht schnelle Aufnahme-Sessions ohne komplexe Einrichtung.
Grundlegende Audio-Editing-Funktionen wie Normalisierung, Rauschunterdrückung und \gls{eq} sind integriert.

\subsection{Audioformat-Spezifikationen}
Für die Audioaufnahme werden folgende technische Parameter festgelegt: Die Sample Rate beträgt 48 kHz, das Format ist \gls{flac}, die Kanäle sind Stereo (2.0) und die Bit Depth beträgt 24 Bit.
Die Begründung der Parameter lautet wie folgt: Diese Sample Rate wird als Standard für Videoproduktionen verwendet und gewährleistet optimale Kompatibilität mit dem Videomaterial \cite{ITU03}.
Die verlustfreie Kompression erhält die vollständige Audioqualität, während gleichzeitig Speicherplatz eingespart wird.
Dies ist besonders wichtig für die spätere Bearbeitung und Mixing-Prozesse.
Eine Stereo-Konfiguration wird für Voice-Over-Aufnahmen als ausreichend erachtet und bietet gute räumliche Präsenz.
Die höhere Bit-Tiefe gegenüber Standard-16-Bit bietet einen größeren Dynamikumfang und mehr Headroom für die Nachbearbeitung.

\section{Kollaborations-Infrastruktur}

\subsection{Datenspeicherung}
Für den gemeinsamen Zugriff auf Projekt-Assets wird ein \gls{nas}-Speicher eingesetzt.
Die Vorteile umfassen zentralisierte Speicherung aller Audio- und Videodateien, gleichzeitigen Zugriff für alle Teammitglieder, automatische Backup-Möglichkeiten, Versionskontrolle bei größeren Dateien und die Vermeidung von Duplizierung großer Mediendateien.

\subsection{Skript-Entwicklung}
Für die kollaborative Erstellung des Drehbuchs wird Google Docs verwendet.
Die Begründung umfasst folgende Aspekte: Mehrere Personen können gleichzeitig am Skript arbeiten (Echtzeit-Kollaboration).
Feedback kann direkt im Dokument gegeben werden (Kommentarfunktion).
Änderungen sind nachvollziehbar und können rückgängig gemacht werden (Versionsverlauf).
Der Zugriff von verschiedenen Geräten ist möglich (Plattformunabhängigkeit).
Keine lokale Softwareinstallation ist erforderlich.

\section{Workflow-Konzept}
Der technische Workflow wird in folgenden Schritten strukturiert.
Zunächst erfolgt die Skript-Entwicklung durch gemeinsame Erstellung in Google Docs.
Anschließend wird die Gameplay-Aufnahme im \gls{mkv}-Format durchgeführt und auf \gls{nas} hochgeladen.
Die Voice-Over-Aufnahme wird in Audacity mit 48kHz/24Bit \gls{flac} durchgeführt.
Bei der Audio-Bearbeitung werden Noise Reduction und Normalisierung angewendet sowie der Export als \gls{flac} vorgenommen.
Beim Video-Assembly werden alle Assets in DaVinci Resolve importiert.
Der Schnitt basiert auf den Drehbuch-Timings als Rohschnitt.
Bei der Audio-Sync wird die Synchronisation von Voice-Over und Gameplay durchgeführt.
Effekte und Overlays werden durch das Hinzufügen von Text-Overlays und Transitions realisiert.
Falls erforderlich, wird beim Color Grading eine Anpassung der Farbgebung vorgenommen.
Abschließend erfolgt der finale Export in hochauflösendem Format.

\section{Qualitätssicherung}
Zur Sicherstellung einer hohen Produktionsqualität werden folgende Maßnahmen implementiert.
Bezüglich der Audio-Qualität wird vor jeder Aufnahme ein Audio-Test durchgeführt, um Pegel und Rauschunterdrückung zu überprüfen.
Bezüglich der Video-Qualität wird Gameplay-Material auf ausreichende Framerate und Auflösung geprüft.
Die Backup-Strategie sieht vor, dass alle Projektdateien redundant auf dem \gls{nas} und lokal gespeichert werden.
Der Review-Prozess stellt sicher, dass vor dem finalen Export das Video von mehreren Teammitgliedern gesichtet wird.

\chapter{Screendumps vom Ergebnis}\label{ch:screendumps}

In diesem Kapitel werden ausgewählte Screendumps des finalen Video-Reviews präsentiert, die zentrale visuelle Aspekte der Produktion dokumentieren.

\section{Intro-Sequenz}

Die Intro-Sequenz stellt den ersten visuellen Kontakt mit dem Zuschauer her und setzt den Ton für das gesamte Review.

% Platzhalter für Screendump der Intro-Sequenz
% \begin{figure}[H]
%     \centering
%     \includegraphics[width=0.9\textwidth]{bilder/screendump_intro.png}
%     \caption{Intro-Sequenz des Video-Reviews (0:00-0:20)}
%     \label{fig:screendump_intro}
% \end{figure}

In diesem Screendump wird der energetische Einstieg visualisiert, bei dem das Spiel \textit{White Lavender} erstmals präsentiert wird. Die visuelle Gestaltung kommuniziert bereits die Comedy-Ausrichtung des Reviews.

\section{Genre-Erklärung mit B-Roll}

Während der theoretischen Erklärung des Soulslike-Genres werden etablierte Titel als Referenz gezeigt.

% Platzhalter für Screendump der Genre-Erklärung
% \begin{figure}[H]
%     \centering
%     \includegraphics[width=0.9\textwidth]{bilder/screendump_genre.png}
%     \caption{B-Roll-Material während der Genre-Erklärung}
%     \label{fig:screendump_genre}
% \end{figure}

Dieser Screendump zeigt die Integration von B-Roll-Material aus bekannten Soulslike-Titeln, die als visueller Kontext für die Erklärung dienen.

\section{Grafikstil-Präsentation}

Ein zentraler Aspekt des Reviews ist die Präsentation des einzigartigen visuellen Stils von \textit{White Lavender}.

% Platzhalter für Screendump des Grafikstils
% \begin{figure}[H]
%     \centering
%     \includegraphics[width=0.9\textwidth]{bilder/screendump_grafik.png}
%     \caption{Präsentation des retro-artigen Grafikstils mit CRT-Filter}
%     \label{fig:screendump_grafik}
% \end{figure}

In diesem Screendump wird der charakteristische retro-artige Stil mit CRT-Filter sichtbar, der das Spiel visuell von anderen Genre-Vertretern abhebt.

\section{Movement-Showcase}

Die Demonstration der Movement-Mechaniken visualisiert eines der diskutierten Gameplay-Elemente.

% Platzhalter für Screendump des Movement-Showcases
% \begin{figure}[H]
%     \centering
%     \includegraphics[width=0.9\textwidth]{bilder/screendump_movement.png}
%     \caption{Movement-Showcase mit Fokus auf Charakter-Animationen}
%     \label{fig:screendump_movement}
% \end{figure}

Dieser Screendump dokumentiert die Präsentation der Bewegungsmechanik, einschließlich der kritisch angemerkten „klebe-am-Boden"-Effekte.

\section{Combat-Szenen}

Die Kampfszenen gegen normale Gegner demonstrieren die Core-Gameplay-Loop.

% Platzhalter für Screendump einer Combat-Szene
% \begin{figure}[H]
%     \centering
%     \includegraphics[width=0.9\textwidth]{bilder/screendump_combat.png}
%     \caption{Combat gegen normale Gegner mit verschiedenen Waffen}
%     \label{fig:screendump_combat}
% \end{figure}

In diesem Screendump werden Kampfsequenzen gezeigt, die sowohl die Waffenvielfalt als auch die diskutierten Probleme wie Hitbox-Issues visualisieren.

\section{Boss-Kämpfe}

Boss-Kämpfe stellen das Herzstück des Soulslike-Genres dar und werden entsprechend prominent präsentiert.

% Platzhalter für Screendump eines Boss-Kampfes
% \begin{figure}[H]
%     \centering
%     \includegraphics[width=0.9\textwidth]{bilder/screendump_boss.png}
%     \caption{Boss-Kampf-Szene mit kreativem Boss-Design}
%     \label{fig:screendump_boss}
% \end{figure}

Dieser Screendump dokumentiert einen der Boss-Kämpfe und illustriert das kreative Design der Gegner.

\section{Endboss-Bug-Demonstration}

Ein kritischer Aspekt des Reviews ist die Demonstration technischer Probleme, insbesondere des Endboss-Bugs.

% Platzhalter für Screendump des Endboss-Bugs
% \begin{figure}[H]
%     \centering
%     \includegraphics[width=0.9\textwidth]{bilder/screendump_endboss_bug.png}
%     \caption{Demonstration des Endboss-Bugs (Charakter steht unter dem Boss)}
%     \label{fig:screendump_endboss_bug}
% \end{figure}

In diesem Screendump wird der diskutierte Bug visualisiert, bei dem der Endboss den Spieler nicht treffen kann, wenn dieser direkt unter ihm steht.

\section{Open World Exploration}

Die Exploration der Spielwelt stellt einen positiv bewerteten Aspekt dar.

% Platzhalter für Screendump der Exploration
% \begin{figure}[H]
%     \centering
%     \includegraphics[width=0.9\textwidth]{bilder/screendump_exploration.png}
%     \caption{Open World Exploration mit Portal-System}
%     \label{fig:screendump_exploration}
% \end{figure}

Dieser Screendump zeigt die verschiedenen Gebiete und das Portal-basierte Reisesystem der Spielwelt.

\section{NPC-Interaktionen}

Die Charakter-Interaktionen werden als positiver Aspekt des Spiels hervorgehoben.

% Platzhalter für Screendump einer NPC-Interaktion
% \begin{figure}[H]
%     \centering
%     \includegraphics[width=0.9\textwidth]{bilder/screendump_npc.png}
%     \caption{NPC-Dialog mit humorvollem Touch}
%     \label{fig:screendump_npc}
% \end{figure}

In diesem Screendump wird ein NPC-Dialog präsentiert, der die charmante Persönlichkeit der Charaktere illustriert.

\section{Fazit-Sequenz}

Die finale Fazit-Sequenz fasst die Bewertung zusammen und präsentiert die Gesamt-Einschätzung.

% Platzhalter für Screendump der Fazit-Sequenz
% \begin{figure}[H]
%     \centering
%     \includegraphics[width=0.9\textwidth]{bilder/screendump_fazit.png}
%     \caption{Fazit-Sequenz}
%     \label{fig:screendump_fazit}
% \end{figure}

In diesem Screendump wird die visuelle Gestaltung des Fazits dokumentiert.

\newpage
\bibliography{literatur}
\bibliographystyle{hwrbib}

% KI-Verzeichnis
\addchap{KI-Verzeichnis}\label{ch:ki}
% die alte Vorlage hat kein KI Verzeichnis
% das hier basiert auf einer Bachelorarbeit aus 2024
% aber wenn ihr sucht, findet ihr bestimmt auch Vorschriften
\begin{table}[H]
    \centering
    \begin{tabular}{|l|l|l|}
         \hline
         \textbf{\acrshort{KI}-basiertes Hilfsmittel} & \textbf{Einsatzform} & \textbf{Betroffene Teile der Arbeit} \\
         \hline
         \hline
         DeepL Translator & Übersetzung von Textpassagen & \autoref{abstract}Abstract\\
         \hline
    \end{tabular}
    \label{tab:ki}
\end{table}

% wenn euer Anhang viele Seiten hat, ergibt es Sinn, zu alphabetischen Seitenzahlen zu "wechseln"
%%%%%%%%%%%%%%%%%%%%%%
% \pagenumbering{Alph}
%%%%%%%%%%%%%%%%%%%%%%

% Ehrenwörtliche Erklärung
\include{kapitel/22erklaerung}

\end{document}
