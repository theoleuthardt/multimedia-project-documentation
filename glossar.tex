% Befehle für Abkürzungen
\newacronym{IT}{IT}{Informatik}
\newacronym{KI}{KI}{Künstliche Intelligenz}
%Eine Abkürzung mit Glossareintrag
\newacronym{RADAR}{RADAR}{Radio Detection And Ranging\protect\glsadd{glos:radar}}

% Befehle für Glossar
\newglossaryentry{glos:glossar}{name=Glossar, description={Ein Glossar umfasst eine Sammlung von Begriffen mit zugehörigen Definitionen und Erklärungen. Diese Begriffe werden im Zusammenhang mit einem Themengebiet verwendet. Die Begriffserläuterungen können spezifisch oder bereichsübergreifend sein. Die Zusammenstellung der Begriffe kann durch die Lernenden selbst erfolgen. \cite{Uni17}}}
\newglossaryentry{glos:radar}{name=Radar, description={Ein aktives Sensorsystem, das Oberflächen-Features auf der Erde erkennt, indem sie mit polarisierten Radiowellen bestrahlt werden und die reflektierte Energie, einschließlich Entfernung, Richtung und Polarisierung, gemessen wird. \cite{Esr25}}}

% wird auch in der alten Vorlage nicht benutzt?
% Befehle für Symbole
%%%%%%%%%%%%%%%%%%%%%
%\newglossaryentry{symb:Pi}{
%name=$\pi$,
%description={Die Kreiszahl.},
%sort=symbolpi, type=symbolslist
%}
%\newglossaryentry{symb:Phi}{
%name=$\varphi$,
%description={Ein beliebiger Winkel.},
%sort=symbolphi, type=symbolslist
%}
%\newglossaryentry{symb:Lambda}{
%name=$\lambda$,
%description={Eine beliebige Zahl, mit der der nachfolgende Ausdruck
%multipliziert wird.},
%sort=symbollambda, type=symbolslist
%}
%%%%%%%%%%%%%%%%%%%%%%